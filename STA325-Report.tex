% Options for packages loaded elsewhere
\PassOptionsToPackage{unicode}{hyperref}
\PassOptionsToPackage{hyphens}{url}
\PassOptionsToPackage{dvipsnames,svgnames,x11names}{xcolor}
%
\documentclass[
  letterpaper,
  DIV=11,
  numbers=noendperiod]{scrartcl}

\usepackage{amsmath,amssymb}
\usepackage{lmodern}
\usepackage{iftex}
\ifPDFTeX
  \usepackage[T1]{fontenc}
  \usepackage[utf8]{inputenc}
  \usepackage{textcomp} % provide euro and other symbols
\else % if luatex or xetex
  \usepackage{unicode-math}
  \defaultfontfeatures{Scale=MatchLowercase}
  \defaultfontfeatures[\rmfamily]{Ligatures=TeX,Scale=1}
\fi
% Use upquote if available, for straight quotes in verbatim environments
\IfFileExists{upquote.sty}{\usepackage{upquote}}{}
\IfFileExists{microtype.sty}{% use microtype if available
  \usepackage[]{microtype}
  \UseMicrotypeSet[protrusion]{basicmath} % disable protrusion for tt fonts
}{}
\makeatletter
\@ifundefined{KOMAClassName}{% if non-KOMA class
  \IfFileExists{parskip.sty}{%
    \usepackage{parskip}
  }{% else
    \setlength{\parindent}{0pt}
    \setlength{\parskip}{6pt plus 2pt minus 1pt}}
}{% if KOMA class
  \KOMAoptions{parskip=half}}
\makeatother
\usepackage{xcolor}
\setlength{\emergencystretch}{3em} % prevent overfull lines
\setcounter{secnumdepth}{-\maxdimen} % remove section numbering
% Make \paragraph and \subparagraph free-standing
\ifx\paragraph\undefined\else
  \let\oldparagraph\paragraph
  \renewcommand{\paragraph}[1]{\oldparagraph{#1}\mbox{}}
\fi
\ifx\subparagraph\undefined\else
  \let\oldsubparagraph\subparagraph
  \renewcommand{\subparagraph}[1]{\oldsubparagraph{#1}\mbox{}}
\fi


\providecommand{\tightlist}{%
  \setlength{\itemsep}{0pt}\setlength{\parskip}{0pt}}\usepackage{longtable,booktabs,array}
\usepackage{calc} % for calculating minipage widths
% Correct order of tables after \paragraph or \subparagraph
\usepackage{etoolbox}
\makeatletter
\patchcmd\longtable{\par}{\if@noskipsec\mbox{}\fi\par}{}{}
\makeatother
% Allow footnotes in longtable head/foot
\IfFileExists{footnotehyper.sty}{\usepackage{footnotehyper}}{\usepackage{footnote}}
\makesavenoteenv{longtable}
\usepackage{graphicx}
\makeatletter
\def\maxwidth{\ifdim\Gin@nat@width>\linewidth\linewidth\else\Gin@nat@width\fi}
\def\maxheight{\ifdim\Gin@nat@height>\textheight\textheight\else\Gin@nat@height\fi}
\makeatother
% Scale images if necessary, so that they will not overflow the page
% margins by default, and it is still possible to overwrite the defaults
% using explicit options in \includegraphics[width, height, ...]{}
\setkeys{Gin}{width=\maxwidth,height=\maxheight,keepaspectratio}
% Set default figure placement to htbp
\makeatletter
\def\fps@figure{htbp}
\makeatother

\KOMAoption{captions}{tableheading}
\makeatletter
\makeatother
\makeatletter
\makeatother
\makeatletter
\@ifpackageloaded{caption}{}{\usepackage{caption}}
\AtBeginDocument{%
\ifdefined\contentsname
  \renewcommand*\contentsname{Table of contents}
\else
  \newcommand\contentsname{Table of contents}
\fi
\ifdefined\listfigurename
  \renewcommand*\listfigurename{List of Figures}
\else
  \newcommand\listfigurename{List of Figures}
\fi
\ifdefined\listtablename
  \renewcommand*\listtablename{List of Tables}
\else
  \newcommand\listtablename{List of Tables}
\fi
\ifdefined\figurename
  \renewcommand*\figurename{Figure}
\else
  \newcommand\figurename{Figure}
\fi
\ifdefined\tablename
  \renewcommand*\tablename{Table}
\else
  \newcommand\tablename{Table}
\fi
}
\@ifpackageloaded{float}{}{\usepackage{float}}
\floatstyle{ruled}
\@ifundefined{c@chapter}{\newfloat{codelisting}{h}{lop}}{\newfloat{codelisting}{h}{lop}[chapter]}
\floatname{codelisting}{Listing}
\newcommand*\listoflistings{\listof{codelisting}{List of Listings}}
\makeatother
\makeatletter
\@ifpackageloaded{caption}{}{\usepackage{caption}}
\@ifpackageloaded{subcaption}{}{\usepackage{subcaption}}
\makeatother
\makeatletter
\@ifpackageloaded{tcolorbox}{}{\usepackage[many]{tcolorbox}}
\makeatother
\makeatletter
\@ifundefined{shadecolor}{\definecolor{shadecolor}{rgb}{.97, .97, .97}}
\makeatother
\makeatletter
\makeatother
\ifLuaTeX
  \usepackage{selnolig}  % disable illegal ligatures
\fi
\IfFileExists{bookmark.sty}{\usepackage{bookmark}}{\usepackage{hyperref}}
\IfFileExists{xurl.sty}{\usepackage{xurl}}{} % add URL line breaks if available
\urlstyle{same} % disable monospaced font for URLs
\hypersetup{
  pdftitle={STA325 Case Study},
  pdfauthor={Chase Mathis, Dillan Sant},
  colorlinks=true,
  linkcolor={blue},
  filecolor={Maroon},
  citecolor={Blue},
  urlcolor={Blue},
  pdfcreator={LaTeX via pandoc}}

\title{STA325 Case Study}
\author{Chase Mathis, Dillan Sant}
\date{}

\begin{document}
\maketitle
\ifdefined\Shaded\renewenvironment{Shaded}{\begin{tcolorbox}[borderline west={3pt}{0pt}{shadecolor}, interior hidden, frame hidden, sharp corners, enhanced, boxrule=0pt, breakable]}{\end{tcolorbox}}\fi

\hypertarget{introduction}{%
\subsection{Introduction}\label{introduction}}

Nobel Prize winning Physicist Richard Feynman said that turbulence was
``the most important unsolved problem of classical physics''. Although,
we still unexpectedly feel bumps on airplines and have trouble truly
predicting where a Hurricane will strike land, we hope to add to the
field of research studying how to model turbulent systems. We position
our research in two key areas:

\begin{enumerate}
\def\labelenumi{\arabic{enumi}.}
\tightlist
\item
  Prediction
\item
  Statistical Inference
\end{enumerate}

Prediction is important in the context of this problem as data from
turbulent systems is very computationally expensive to obtain.
Therefore, experiments using data on the evolution turbulent systems
need better, quicker, cheaper methods to get data.

Statistical Inference is also important as it can help elucidate the
connections between turbulence and initial settings. We hope to build a
model that gives clarity on this as well.

\hypertarget{methodology}{%
\subsection{Methodology}\label{methodology}}

We are given 89 data points, but we only use 66 data points for training
our model as to not overfit our model. We will examine numerous
different types of models including:

\begin{itemize}
\tightlist
\item
  A simple Linear Regression Model
\item
  More Complex Non-Linear Regression Model
\item
  Tree Based Model
\item
  Boosted Tree Model
\end{itemize}

The data has \texttt{4} response variables and \texttt{3} input
variables. Each output variable is a raw moment of a final turbulent
distribution. We've converted the raw moments to central moments so as
for better interpret ability. We keep the first raw moment, however.

As there are only three input variables, we will stray away from methods
which look for sparse models. Our model is, already, sparse so there is
no need to make it more sparse.

\hypertarget{quick-transformations-for-our-input-variables}{%
\paragraph{Quick Transformations for our Input
Variables}\label{quick-transformations-for-our-input-variables}}

\begin{longtable}[]{@{}rrrr@{}}
\toprule()
Reynold's Number & Count & Freyman's Number & Count \\
\midrule()
\endhead
90 & 27 & 0.052 & 23 \\
224 & 26 & 0.300 & 20 \\
398 & 13 & Inf & 23 \\
\bottomrule()
\end{longtable}

Reynold's and Freyman's numbers are non-continuous as there is only
three buckets for the numbers. Therefore I am going to make them a
categorical variable to help interpretability.

\hypertarget{simple-linear-regression-benchmark}{%
\subsubsection{Simple Linear Regression
Benchmark}\label{simple-linear-regression-benchmark}}

\begin{longtable}[]{@{}rrrr@{}}
\caption{Mean Squared Error using Simple Linear
Regression}\tabularnewline
\toprule()
Mean & Variance & Skewness & Kurtosis \\
\midrule()
\endfirsthead
\toprule()
Mean & Variance & Skewness & Kurtosis \\
\midrule()
\endhead
0.0002482 & 33003.25 & 2.128693e+12 & 1.537524e+20 \\
\bottomrule()
\end{longtable}

It is clear that the using a Simple Linear Regression model to predict
anything but the \texttt{mean} is not a good idea as the performance is
not good. Nonetheless it serves as a good baseline for predicting
further complex models.

\hypertarget{polynomials}{%
\subsubsection{Polynomials}\label{polynomials}}

\hypertarget{interaction-effects}{%
\subsubsection{Interaction Effects}\label{interaction-effects}}

\hypertarget{trees}{%
\subsubsection{Trees}\label{trees}}

\hypertarget{boosting}{%
\subsubsection{Boosting}\label{boosting}}

\hypertarget{evaluating-our-models}{%
\subsection{Evaluating Our Models}\label{evaluating-our-models}}

\begin{verbatim}
[1] "Tree Based Model"
\end{verbatim}

\begin{verbatim}
[1] "Tree Based Model"
\end{verbatim}

\begin{verbatim}
[1] "Tree Based Model"
\end{verbatim}

\begin{verbatim}
[1] "Tree Based Model"
\end{verbatim}

The final model selected for predicting all four moments is our
tree-based model. This is the model that has the lowest MSE for mean,
variance, skew, and kurtosis. The tree-based model makes for
interpretable model for a tricky set of data which included only one
quantitative predictor (Stokes number) and two factor variables
(Reynolds number and Frouds number). Our only fear is that this
tree-based model might be prone to overfitting to the training data, and
perhaps would have a high test error. However, we believe that due to
the simplicity of the training data combined with having the lowest
training MSE relative to our other models that the tree-based model
serves as the best predictor of the first four central moments of
particle cluster volume.



\end{document}
